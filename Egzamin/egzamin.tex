\documentclass[11pt]{article}
\usepackage{polski}
\usepackage[utf8]{inputenc}
\usepackage[T1]{fontenc}
\usepackage{graphicx}
\usepackage{grffile}
\usepackage{longtable}
\usepackage{hyperref}
\usepackage{wrapfig}
\usepackage{rotating}
\usepackage[normalem]{ulem}
\usepackage{amsmath}
\usepackage{textcomp}
\usepackage{amssymb}
\usepackage{capt-of}
\usepackage{mathtools}
\usepackage{geometry}
\usepackage{tabto}
\usepackage{listings}

 \geometry{
 a4paper,
 total={170mm,257mm},
 left=20mm,
 top=20mm,
 }

 \hypersetup{
    colorlinks=true,
    linkcolor=blue,
    filecolor=magenta,
    urlcolor=cyan,
}

\title{\textbf{Raport (SNE) Python}}
\author{Wojciech Kubiak}
\date{\today}

\begin{document}
\maketitle
\section{Metoda gradientu}
Zaimplementuj poniżej podany Algorytm Metody Gradientu. Za pomocą tego
algorytmu zbadaj lokalne i globalne minimum następujących funkcji.

\subsection{Implementacja}
\begin{lstlisting}
    # Python
    # Gradient descent
    import random
    
    # Function return true if summed difference between x_new and x_old is grater than epsilon
    # else return false
    def check_epsilon(x_new, x_old, epsilon):
        sum = 0.0
        for i in range(len(x_new)):
            sum += abs(x_new[i] - x_old[i])
        if sum < epsilon:
            return False
        return True
    
    # Function to calculate value from first input function
    def first_function(x_1, x_2):
        return 2*x_1**2 + x_2**2 - 2 * x_1 * x_2 - 2 * x_1 + 1
    
    # Function to calculate first gradient
    def calculate_first_gradient(x_old, epsilon, c):
        print("First gradient")
        print(f"Values: [{x_old[0], x_old[1]}]")
        # function we are calculating gradient for
        print("f(x_1, x_2) = 2x_1^2 + x_2^2 - 2x_1 * x_2 - 2x_1 + 1")
        x_new = list(x_old)  # copy x_old to x_new
        flag = True
        while(flag):
            # assign x_old elements to variables for brevity
            x = x_old[0]
            y = x_old[1]
            x_new[0] = x - c * (4 * x - 2 * y - 2) # derivative by x1
            x_new[1] = y - c * (2 * y - 2 * x) # derivative by x2
            flag = check_epsilon(x_new, x_old, epsilon) # check diference
            x_old = list(x_new)  # copy x_new to x_old
        print(f"Point({x_new[0]}, {x_new[1]})")
        print(f"Value: {first_function(x_new[0], x_new[1])}\n")
    
    # Function to calculate value from second input function
    def second_function(x_1, x_2):
        # x ** y => x to the power of y
        return x_1**4/2 - x_1**3/3 - x_1**2/2 + x_2**2 - 2*x_2 + 1
    
    # Function to calculate second gradient
    def calculate_second_gradient(x_old, epsilon, c):
        print("Second gradient")
        print(f"Values: [{x_old[0], x_old[1]}]")
        print("f(x_1, x_2) = x_1^4/2 - x_1^3/3 - x_1^2/2 + x_2^2 - 2x_2 + 1\n")
        flag = True
        while(flag):
            # assign x_old elements to variables for brevity
            x = x_old[0]
            y = x_old[1]
            x_new = list(x_old)  # copy x_old to x_new
            x_new[0] = x - c * (2 * x ** 3 - x ** 2 - x) # derivative by x1
            x_new[1] = y - c * (2 * y - 4) # derivative by x2
            flag = check_epsilon(x_new, x_old, epsilon) # derivative by x2
            x_old = list(x_new)  # copy x_new to x_old
        print(f"Point({x_new[0]}, {x_new[1]})")
        print(f"Value: {second_function(x_new[0], x_new[1])}\n")
    
    
    def main():
        print("Gradient descent\n")
        # Predefine epsilon and c constant
        epsilon = 0.00001
        c = 0.01
        # Generate list of length 2 with random floats between -5 <= x < 5
        input_first = [round(random.uniform(-5, 5), 2) for i in range(2)]
        calculate_first_gradient(input_first, epsilon, c)
        # Generate list of length 2 with random floats between -3 <= x < 3
        input_second = [round(random.uniform(-2, 2), 2) for i in range(2)]
        calculate_second_gradient(input_second, epsilon, c)
    
    
    if __name__ == "__main__":
        main()
    
    \end{lstlisting}
    \subsection{Wynik uruchomienia}
    \begin{lstlisting}
        Wojciechs-Mac-Pro:Python wojciechkubiak$ python3 gradient.py 
Gradient descent

First gradient
Values: [(-0.25, 2.87)]
f(x_1, x_2) = 2x_1^2 + x_2^2 - 2x_1 * x_2 - 2x_1 + 1
Point(1.0004937371983043, 1.0007988835683663)
Value: 3.368907282030875e-07

Second gradient
Values: [(-0.18, -1.54)]
f(x_1, x_2) = x_1^4/2 - x_1^3/3 - x_1^2/2 + x_2^2 - 2x_2 + 1

Point(-0.4994828268443053, 1.9999030466941128)
Value: 0.94772296987149

Wojciechs-Mac-Pro:Python wojciechkubiak$ python3 gradient.py 
Gradient descent

First gradient
Values: [(1.7, -3.8)]
f(x_1, x_2) = 2x_1^2 + x_2^2 - 2x_1 * x_2 - 2x_1 + 1
Point(0.9995058630131678, 0.9992004695602072)
Value: 3.3743652272377744e-07

Second gradient
Values: [(0.56, -1.6)]
f(x_1, x_2) = x_1^4/2 - x_1^3/3 - x_1^2/2 + x_2^2 - 2x_2 + 1

Point(0.9999981084405191, 1.9995135906874189)
Value: 0.665694084640891
    \end{lstlisting}
\subsection{Analiza działania programu}
Losujemy punkt K(x1, x2) i sprawdzamy pochodna w punkcie K aby określić nachylenie funkcji. Następnie dopasowujemy parametry w taki sposób aby przesunąć je w stronę minimum lokalnego. Od parametru odejmujemy stałą c (tempo uczenia) pomnożoną przez wartość pochodnej w danym punkcie z parametru.
Działanie programu powtarzamy dopóki różnica między nowymi a starymi parametrami nie będzie mniejsza od epsilona. Tępo uczenia nie może być za duże ponieważ możemy pominąć minumum lokalne przy kolejnej iteracji.

\section{Maszyna Boltzmanna (MB)}
Zaimplementuj poniżej podany Algorytm MB. Zbadać należność zachowania MB od stałej
tempetatury T > 0 (Zob. dwie Uwagi zaraz przed i po Twierdzeniem 6.1.1, Notatki 6).
Lepiej byłoby, wyniki przestawione $0 \rightarrow \_$ i $0 \rightarrow *$.

\subsection{Implementacja}
\begin{lstlisting}
    # Python
    import random
    
    euler = 2.718281828459
    
    
    def main():
        temperatures = [0.01, 0.1, 1, 3, 10]
    
        z_list = [1 if x < 10 else 0 for x in range(20)]
    
        vec_c = get_c_vector(z_list)
        vec_w = get_w_vector(vec_c)
        theta = get_theta(vec_c)
    
        for temperature in temperatures:
    
            print("Current temperature: {}".format(temperature))
            x = [[] for _ in range(11)]
            x[0] = gen_random_vector()
            t = 0
    
            while(t < 10):
    
                beta = gen_beta_vector()
                for i in range(20):
    
                    if(beta[i] >= 0 and beta[i] <= f_uit(vec_w[i], x[t], theta[i], temperature)):
                        x[t+1].append(1)
                    elif(beta[i] <= 1 and beta[i] >= f_uit(vec_w[i], x[t], theta[i], temperature)):
                        x[t+1].append(0)
    
                t += 1
                print_vector(x[t])
    
    
    def f_uit(w, x, theta, temperature):
    
        s = 0
        uit = 0
        for j in range(20):
            s += w[j]*x[j]
            uit = s - theta
    
        return 1 / (1 + euler ** (-(uit/temperature)))
    
    
    def get_theta(vec):
        ''' Return theta value for vector'''
        sum = [0]*20
        for i in range(20):
            for j in range(20):
                sum[i] += vec[i][j]
        return sum
    
    
    def get_w_vector(vec):
        ''' Compute vector "w" from vector "c"'''
        w = [[] for _ in range(len(vec))]
        for i in range(20):
            for j in range(20):
                w[i].append(2*vec[i][j])
        return w
    
    
    def get_c_vector(vec):
        '''Generate "c" vector from "z" list '''
        c = [[] for _ in range(len(vec))]
        for i in range(len(vec)):
            for j in range(len(vec)):
                if i != j:
                    c[i].append((vec[i] - 0.5) * (vec[j] - 0.5))
                else:
                    c[i].append(0.0)
        return c
    
    
    def gen_random_vector():
        '''Generate random list, each element is 0 or 1.'''
        return [random.randint(0, 1) for x in range(20)]
    
    
    def gen_beta_vector():
        '''Generate random list, each element is range from 0 to 1.'''
        return [random.random() for x in range(20)]
    
    
    def print_vector(vec):
        for i in range(len(vec)):
            if (vec[i] <= 0.0):
                print(" _ ", end='')
            else:
                print(" * ", end='')
        print(" ")
    
    
    if __name__ == "__main__":
        main()
    
    
    
    \end{lstlisting}
    \subsection{Wynik uruchomienia}
    \begin{lstlisting}
        Wojciechs-Mac-Pro:Python wojciechkubiak$ python3 boltzman.py 
        Current temperature: 0.01
         *  *  *  *  *  *  *  *  *  *  _  _  _  _  _  _  _  _  _  _  
         *  *  *  *  *  *  *  *  *  *  _  _  _  _  _  _  _  _  _  _  
         *  *  *  *  *  *  *  *  *  *  _  _  _  _  _  _  _  _  _  _  
         *  *  *  *  *  *  *  *  *  *  _  _  _  _  _  _  _  _  _  _  
         *  *  *  *  *  *  *  *  *  *  _  _  _  _  _  _  _  _  _  _  
         *  *  *  *  *  *  *  *  *  *  _  _  _  _  _  _  _  _  _  _  
         *  *  *  *  *  *  *  *  *  *  _  _  _  _  _  _  _  _  _  _  
         *  *  *  *  *  *  *  *  *  *  _  _  _  _  _  _  _  _  _  _  
         *  *  *  *  *  *  *  *  *  *  _  _  _  _  _  _  _  _  _  _  
         *  *  *  *  *  *  *  *  *  *  _  _  _  _  _  _  _  _  _  _  
        Current temperature: 0.1
         *  _  _  _  *  *  *  *  _  _  *  _  *  _  *  _  _  _  _  *  
         *  *  *  *  *  *  *  *  *  *  _  _  _  _  _  _  _  _  _  _  
         *  *  *  *  *  *  *  *  *  *  _  _  _  _  _  _  _  _  _  _  
         *  *  *  *  *  *  *  *  *  *  _  _  _  _  _  _  _  _  _  _  
         *  *  *  *  *  *  *  *  *  *  _  _  _  _  _  _  _  _  _  _  
         *  *  *  *  *  *  *  *  *  *  _  _  _  _  _  _  _  _  _  _  
         *  *  *  *  *  *  *  *  *  *  _  _  _  _  _  _  _  _  _  _  
         *  *  *  *  *  *  *  *  *  *  _  _  _  _  _  _  _  _  _  _  
         *  *  *  *  *  *  *  *  *  *  _  _  _  _  _  _  _  _  _  _  
         *  *  *  *  *  *  *  *  *  *  _  _  _  _  _  _  _  _  _  _  
        Current temperature: 1
         _  _  _  *  _  _  _  _  _  _  *  *  *  *  *  *  *  _  *  *  
         _  *  _  _  _  _  _  _  _  _  *  *  *  *  *  *  *  *  *  *  
         _  _  _  _  _  _  _  _  _  _  *  *  *  *  *  *  *  *  *  *  
         _  _  _  _  _  _  _  _  _  _  *  *  *  *  *  *  *  *  *  *  
         _  _  _  _  _  _  _  _  _  _  *  *  *  *  *  *  *  *  *  *  
         _  _  _  _  _  _  _  _  _  _  *  *  *  *  *  *  *  *  *  *  
         _  _  _  _  *  _  _  _  _  _  *  *  *  *  *  *  *  *  *  *  
         _  _  _  _  _  _  _  _  _  _  *  *  *  *  *  *  *  *  *  *  
         _  _  _  _  _  _  _  _  _  _  *  *  *  *  *  *  *  *  *  *  
         _  _  _  _  _  _  _  _  _  _  *  *  *  _  *  *  *  *  *  *  
        Current temperature: 3
         *  *  _  *  *  *  _  _  _  _  *  _  *  *  *  *  *  *  *  *  
         _  _  _  _  _  _  *  *  _  _  _  *  *  *  _  *  _  *  _  *  
         *  _  _  _  _  *  _  _  _  _  *  _  _  _  *  _  *  *  *  _  
         _  _  _  *  *  _  _  _  _  *  _  *  *  *  *  _  *  _  *  *  
         _  *  _  _  *  _  _  _  _  _  _  _  _  *  _  *  _  *  _  *  
         _  *  _  _  *  _  *  _  *  _  _  *  _  *  *  _  *  _  *  *  
         _  *  *  _  _  _  *  *  _  _  _  _  *  *  _  _  _  _  _  *  
         *  _  *  _  _  *  *  *  *  _  *  _  _  _  *  *  _  *  _  _  
         _  *  *  _  *  *  _  _  _  *  _  _  *  *  *  _  _  *  _  _  
         *  *  _  *  _  _  *  *  *  *  _  *  *  _  _  _  _  *  _  _  
        Current temperature: 10
         _  *  _  *  *  _  *  *  *  _  _  _  *  *  *  _  _  *  *  _  
         *  _  *  _  _  *  *  *  _  _  _  *  *  *  *  *  _  _  _  _  
         *  _  _  *  _  *  _  *  _  _  _  _  *  *  _  _  *  *  *  *  
         *  *  _  _  _  *  *  *  *  *  _  _  *  _  _  _  _  *  *  _  
         *  *  _  *  _  *  *  _  *  _  *  _  _  _  *  _  _  _  *  *  
         *  *  *  _  _  _  _  _  *  _  _  _  *  _  _  *  *  _  _  *  
         _  _  *  *  *  *  *  _  *  _  _  _  *  _  *  _  _  *  *  _  
         *  _  *  _  *  *  *  _  *  *  _  *  _  *  _  *  *  *  _  _  
         *  _  _  _  *  *  _  _  _  *  _  *  *  _  *  _  *  _  _  _  
         *  _  _  *  _  *  _  *  *  _  _  _  _  _  *  *  _  _  _  *  
        Wojciechs-Mac-Pro:Python wojciechkubiak$ 
    \end{lstlisting}
\subsection{Analiza działania programu}
Stany Maszyny Boltzmana obliczamy za pomocą funkcji

\[
    x_i(t+1) = 
    \begin{cases}
        1, & \text{gdy $0 \leq \beta_i \leq f(u_i(t))$} \\
        0, & \text{gdy $ f(u_i(t)) \leq \beta_i \leq 1$}
    \end{cases}
\]
gdzie $\beta_i \in [0, 1]$, funkcja $f(u_i(t))$ wygląda następująco
\[f(u_i) = \frac{1}{1 + e^{-\frac{u_i(t)}{T}}}\]

W naszym przypadku $u_i(t)$ obliczamy z
\[u_i(t) = \bigg\{ \sum^{20}_{j=1}w_{ij}x_j(t) \bigg\} - \theta_i\]

dla wartość 0.25 dla $x_j = 0$ oraz -0.25 dla $x_j = 1$.
Dla temperatur bliskich zero otrzymamy
\[lim_{T \rightarrow 0}f(0.25) = lim_{T \rightarrow 0}\frac{1}{1 + e^{-\frac{0.25}{T}}} = lim_{x \rightarrow - \infty}\frac{1}{1 + e^x} = lim_{x \rightarrow 0} \frac{1}{1 + x} = 1\]
\[lim_{T \rightarrow 0}f(-0.25) = lim_{T \rightarrow 0}\frac{1}{1 + e^{\frac{0.25}{T}}} = lim_{x \rightarrow + \infty}\frac{1}{1 + e^x} = lim_{x \rightarrow + \infty} \frac{1}{1 + x} = 0\]

Dla takich temperatur stan maszyny będzie zmieniał się na przeciwny. Stan każdego neuronu $x_i$ zmieni się w następujący sposób:
\[x_i(t) = 0 \Longrightarrow x_i(t+1) = 1\]

Wartości funkcji $f(u_i)$ zbliżają się do $\frac{1}{2}$ kiedy temperatura rośnie.
\[lim_{T \rightarrow + \infty}f(\pm 0.25) = lim_{T \rightarrow + \infty}\frac{1}{1 + e^{\frac{\pm 0.25}{T}}} = lim_{x \rightarrow \pm 0}\frac{1}{1+e^x} = lim_{x \rightarrow 1}\frac{1}{1 + x} = \frac{1}{2}\]
Kolejne stany maszyny stają się bardziej losowe, gdyż prawdopodobieństwo $P(f(u_i) \leq \beta) \rightarrow \frac{1}{2}$ i $P(\beta \leq f(u_i)) \rightarrow \frac{1}{2}$.

\end{document}